% T I T L E   P A G E
\begin{titlepage}
        \begin{center}
        \vspace*{1.0cm}

        \Huge
        {\bf Explaining embedding results for scoring alignments }

        \large
        Final Progress Report \\

        \vspace*{1.0cm}

        \normalsize
        by \\

        \vspace*{1.0cm}

        \Large
        Riley Gavigan \\

        \vspace*{3.0cm}

        \normalsize
        CS 4490Z \\
        Thesis Supervisor: Lucian Ilie \\ 
        Course Instructor: Nazim Madhavi \\

        \vspace*{2.0cm}

        Department of Computer Science \\
        University of Western Ontario, London, N6A 5B7, Ontario, Canada \\
        \today \\

        \vspace*{1.0cm}
        \end{center}
\end{titlepage}

% Define a new chapter format that doesn't display the chapter number
\titleformat{\chapter}[display]
  {\normalfont\huge\bfseries}{}{0pt}{\Huge}

\setcounter{page}{2}

\cleardoublepage
\phantomsection

\printglossary

% L I S T   O F   A B B R E V I A T I O N S
\renewcommand*{\abbreviationsname}{Abbreviations}
\printglossary[type=abbreviations]
\cleardoublepage
\phantomsection

% A B S T R A C T
% ---------------
\chapter*{Structured Abstract}
\addcontentsline{toc}{chapter}{Structured Abstract}

\subsubsection*{Context and motivation}
The \textit{E}-score protein alignment scoring method \autocite{Ashrafzadeh:2023} outperforms state-of-the-art methods, supported by comparing ProtT5 \autocite{Elnaggar:2021} \textit{E}-score results with BLOSUM45 \autocite{Henikoff:1992}.

This research aimed to understand \textit{E}-score results, building upon the observation that mean cosine similarity results between two embeddings are not evenly distributed.

By understanding the underlying causes of the observed results, we can improve the \textit{E}-score method. Insights can be used to fine-tune the transformer models \autocite{Elnaggar:2021, Rives:2021} and performance of embeddings.

\subsubsection*{Research questions}
\begin{itemize}
    \item{What properties of embeddings produce better cosine similarity results?}
    \item{Why do cosine similarity results primarily fall within a positive range?}
    \item{How can models be fine-tuned to produce better embeddings?}
\end{itemize}

\subsubsection*{Principal ideas}
Positive cosine similarity results imply the produced embeddings are mostly similar. Comparing different embedding types provides insight into their distributions. Through these comparisons, conclusions about properties that improve \textit{E}-score results were drawn.

\subsubsection*{Research methodology}
This research is a data science investigation to obtain insight about the embeddings and cosine similarity results in the \textit{E}-score method.

\subsubsection*{Anticipated results}
This study primarily aimed to obtain insight and knowledge for the \textit{E}-score method, specifically:
\begin{itemize}
    \item{Knowledge about the distributions of different embedding types}
    \item{Knowledge about the cosine similarity between embeddings}
    \item{Insight to fine-tune and improve models}
\end{itemize}

\subsubsection*{Novelty}
By building upon a novel method for scoring protein alignments using cosine similarity \autocite{Ashrafzadeh:2023}, novel conclusions about embeddings and cosine similarity were made, leading to further research that can improve embeddings and models.

\subsubsection*{Impact}
Improvements in transformer models for the \textit{E}-score alignment scoring method can be made through the insight this research found. Improvements may also be applicable to \gls{NLP} Models such as T5 \autocite{Raffel:2020}.

\subsubsection*{Progress and completed work}
Insight into properties behind embedding type distributions was obtained. From these properties, cosine similarity results were explained. These properties were explained through conducted research and simulation in combination with insight from biochemical background research.

\subsubsection*{Limitations}
No limitations are known to exist in this research.

\cleardoublepage
\phantomsection

% T A B L E   O F   C O N T E N T S
% ---------------------------------
\renewcommand\contentsname{Table of Contents}
\tableofcontents
\cleardoublepage
\phantomsection

% L I S T   O F   S Y M B O L S
% ---------------------------
\cleardoublepage
\phantomsection

% Change page numbering back to Arabic numerals
\pagenumbering{arabic}
